\documentclass{standalone}
\usepackage[latin1]{inputenc} 
\usepackage[spanish,es-lcroman]{babel}                                          
\selectlanguage{spanish}    
\usepackage{tikz-3dplot-circleofsphere}
\usetikzlibrary{3d,backgrounds,quotes,angles,decorations.text,
  shapes.geometric, shapes.misc,decorations.shapes, arrows.meta, babel}
\usepackage{wasysym}
\usepackage{marvosym}
    
\begin{document}
  \centering
  \tdplotsetmaincoords{70}{130}
  \begin{tikzpicture}[tdplot_main_coords, scale=2]

    \pgfmathsetmacro{\r}{2} % radio
    \pgfmathsetmacro{\rt}{.4} % radio de la Tierra
    \pgfmathsetmacro{\colatitud}{55}%{55}
    \pgfmathsetmacro{\decsol}{-10}%{55}
    \coordinate (O) at (0,0,0); % centro

    \shade[ball color = blue, opacity = 0.2,tdplot_screen_coords] (O)
    circle [radius = \r*1cm];

    
%    meridianos celestes (¡lo que me costó!)
    \tdplotsetrotatedcoords{0}{\colatitud}{0}         
    \begin{scope}
      \foreach \a in {0,30,...,360} {
        \tdplottransformmainrot{sin(\colatitud)*cos(\a)}
        {sin(\colatitud)*sin(\a)}{0}
        \tdplotgetpolarcoords{\tdplotresx}{\tdplotresy}{\tdplotresz}
        \tdplotCsDrawGreatCircle[thin,red!60!white,
        opacity=.7]{\r}{\tdplotresphi}{\tdplotrestheta} }
    \end{scope}


    \tdplotsetrotatedcoords{0}{-\colatitud}{0}

    
    % solcito
    % \foreach \a in {50} { \shade [ball color=yellow,
    %   opacity=.8,tdplot_rotated_coords]
    %   ({.99*\r*cos(\a)*sin(90+\decsol)},{.99*\r*sin(\a)*sin(90+\decsol)},
    %   {.99*\r*cos(90+\decsol)}) circle (2pt); }

   


    % estrellas
    
    
    %paralelos celestes
    \foreach \a in {-75,-60,...,75}
    {\tdplotCsDrawCircle[red!60!white,thin,opacity=.7]
      {\r}{0}{-\colatitud}{\a}}

    
    % paralelo solar
    % \tdplotCsDrawCircle[yellow!80!black,ultra
    % thick,opacity=1,tdplotCsFill/.style={opacity=0.2}]
    % {\r}{0}{-\colatitud}{-\decsol}


    % círculos verticales
    % \foreach \a in {0,30,...,360} {\tdplotCsDrawLonCircle[very
    %   thin, red!60!white, opacity=.7]{\r}{\a}}

    % almucantaradas
    % \foreach \a in {-75,-60,...,75} {\tdplotCsDrawLatCircle[very
    %   thin,red!60!white, opacity=.7]{\r}{\a}}
       
    % horizonte
    \tdplotCsDrawLatCircle[thin,green!80!black,
    tdplotCsFill/.style={opacity=0.2}]{\r}{0}

    % ecuador terrestre
    % \tdplotCsDrawGreatCircle%
    % [red,thin,tdplotCsFill/.style={opacity=0.1}]{.102mm}{0}{-\colatitud}

    % ecuador celeste
    \tdplotCsDrawGreatCircle%
    [yellow!70!white,thick,
    tdplotCsFill/.style={opacity=0.1}]{\r}{0}{-\colatitud}

    % etiquetas
    % \fill[green!60!black] (0,0,\r) circle (1pt) node [above
    % left=2pt]{Cenit ($Z$)}; \fill[green!60!black] (0,0,-\r) circle
    % (1pt) node [below right=2pt]{Nadir};

    % \draw[decoration={text along
    %   path,text={|\sffamily\footnotesize\color{red!80!white}|Ecuador
    %     celeste}, raise=5pt}, decorate, tdplot_rotated_coords]
    % plot[variable=\t,domain=45:170] ({\r*cos(\t)},{\r*sin(\t)},0);

    % \draw[decoration={text along
    %   path,text={|\sffamily\footnotesize\color{cyan!80!black}|Meridiano
    %     del lugar}, raise=3pt}, decorate, tdplot_rotated_coords]
    % plot[variable=\t,domain=-60:60] ({\r*cos(\t)},0,{\r*sin(\t)});
    
    \draw[decoration={text along path,
      text={|\sffamily\footnotesize\color{green!60!black}|Horizonte},
      raise=3pt}, decorate, tdplot_main_coords]
    plot[variable=\t,domain=50:190] ({\r*cos(\t)},{\r*sin(\t)},0);

    \filldraw[red!70!black, opacity=.7,tdplot_rotated_coords ] (xyz
    spherical cs:radius=\r,latitude=90,longitude=0) circle(1pt) node
    [above right=3pt] {$PSC$} (xyz spherical
    cs:radius=\r,latitude=-90,longitude=0) circle (1pt) node [below
    left=3pt] {$PNC$};

    \begin{scope}[on background layer]

      % vertical del lugar
      % \draw[ultra thick, green!60!black] (0,0,-\r) --
      % node[sloped,pos=.75,above]{\footnotesize vertical del lugar}
      % (0,0,\r);
      
      % eje del mundo
      \filldraw[very thick,red!70!black,
      opacity=.7,tdplot_rotated_coords] (xyz spherical
      cs:radius=\r,latitude=-90,longitude=0) --
      node[sloped,pos=.2,above,red!80!black, opacity=1]{\footnotesize
        eje del mundo} (xyz spherical
      cs:radius=\r,latitude=90,longitude=0);
      
      \shade [ball color=blue, opacity=.8] (O) circle[radius=2mm]
      node[below right=4mm] {};

      % \draw[red!90!black] (180:6mm) node [text width=10mm] {\tiny
      % Ecuador Terrestre};

       \draw[tdplot_rotated_coords] (xyz spherical
       cs:radius=.09mm,latitude=90-\colatitud,longitude=90) node
       []{\small \Gentsroom};

      %% sentido de rotación

      % Declinaci�n


      
    \end{scope}

      \tdplotsetrotatedcoords{0}{-\colatitud}{30}
      
      % \draw[green!70!black,>=Latex,ultra thick, tdplot_rotated_coords]
      % plot[variable=\t,domain=0:30] (xyz spherical
      % cs:radius=\r,latitude=\t,longitude=120);

      % \draw[green!70!black,>=Latex,ultra thick, tdplot_rotated_coords,
      % fill=green!70!black, opacity=.6] plot[variable=\t,domain=0:30]
      % (xyz spherical cs:radius=\r,latitude=\t,longitude=120) --
      % (0,0,0) -- cycle;

      % \draw[green!50!black,>=Latex, thick, tdplot_rotated_coords, ->]
      % plot[variable=\t,domain=0:30] (xyz spherical
      % cs:radius=.85*\r,latitude=\t,longitude=120);

      % \draw[green!50!black, tdplot_rotated_coords] (xyz spherical
      % cs:radius=.7*\r,latitude=18,longitude=120) node {\Large
      %   $\delta$};
      
      % % Ascensi�n Recta
      % \draw[red!70!black,>=Latex,ultra thick, tdplot_rotated_coords]
      % plot[variable=\t,domain=90:120] (xyz spherical
      % cs:radius=\r,latitude=0,longitude=\t);

      % \draw[red!80!white,thick, tdplot_rotated_coords, opacity=.6,
      % fill=red!70!white] plot[variable=\t,domain=0:-30]
      % ({\r*cos(\t)},{\r*sin(\t)},0) -- (0,0,0) -- cycle;

      % \draw[red!70!black,>=Latex, thick, tdplot_rotated_coords,
      % ->] plot[variable=\t,domain=90:120] (xyz spherical
      % cs:radius=.85*\r,latitude=0,longitude=\t);
      
      % \draw[red!70!black, tdplot_rotated_coords] (xyz spherical
      % cs:radius=.65*\r,latitude=0,longitude={90+18}) node {\Large
      %   $\alpha$};


    \tdplotsetrotatedcoords{0}{-\colatitud}{0}

    % meridiano del lugar
    % \tdplotCsDrawLonCircle[very
    % thick,cyan!80!white,tdplotCsFill/.style={opacity=0.2}]{\r}{90}
    
    % % ecl�ptica
    % \tdplotCsDrawGreatCircle%
    % [yellow,thin,tdplotCsFill/.style={opacity=0.2}]
    % {\r}{20}{-\colatitud-20}

%    \tdplotsetrotatedcoords{20}{-\colatitud-20}{0}

    \shade [ball color=yellow, opacity=.8, tdplot_rotated_coords] (xyz
    spherical cs:radius=1*\r,latitude=0,longitude=270) circle (2pt)
    node[below=3pt] {\LARGE \bfseries Culminaci�n inferior (0h)};
        
    % \draw[yellow!80!black,>=Latex, ultra thick, tdplot_rotated_coords,
    % ->] plot[variable=\t,domain=106:116] (xyz spherical
    % cs:radius=1*\r,latitude=0,longitude=\t);

    % \draw[decoration={text along path,
    %   text={|\sffamily\footnotesize\color{yellow!85!white}|Ecl�ptica},
    %   raise=3pt}, decorate, tdplot_rotated_coords]
    % plot[variable=\t,domain=70:-90] (xyz spherical
    % cs:radius=1*\r,latitude=0,longitude=\t);
    
 %   \tdplotsetrotatedcoords{0}{-\colatitud}{30}

 %   mov sol
    %     \draw[yellow,dashed, ultra thick, tdplot_rotated_coords, ]
    % plot[variable=\t,domain=170:140] (xyz spherical
    % cs:radius=1*\r,latitude=0,longitude=\t);

    % \draw[yellow,>=Latex, thick, tdplot_rotated_coords]
    % plot[variable=\t,domain=0:360] (xyz spherical
    % cs:radius=1*\r,latitude=0,longitude=\t);

    % \draw[decoration={text along path,
    %   text={|\sffamily\footnotesize\color{yellow!55!black}|Movimiento
    %     aparente del Sol}, raise=3pt}, decorate,
    % tdplot_rotated_coords] plot[variable=\t,domain=100:10] (xyz
    % spherical cs:radius=1*\r,latitude=0,longitude=\t);



    % Estrella
    % \draw[decoration={shape backgrounds,shape=star,shape size=5pt},
    % decorate, tdplot_rotated_coords,fill=purple] (xyz spherical
    % cs:radius=\r,latitude=30,longitude={90+30});

    % \draw[purple,>=Latex, ultra thick, tdplot_rotated_coords, ->]
    % plot[variable=\t,domain=115:90] (xyz spherical
    % cs:radius=1*\r,latitude=30,longitude=\t);

    % \foreach \a in {0,10,...,360} {

    %   \shade [ball color=yellow, opacity=.8]
    %   ({.99*\r*cos(\a)},{.99*\r*sin(\a)},0) circle (1pt);

    %   }
    
    % Punto Vernal
    % \draw[tdplot_rotated_coords, fill=black] (\r,0,0) circle (.7pt)
    % node [above right] {\Large $\vernal$};
    
    % \draw[>=Latex, ultra thick, tdplot_rotated_coords, ->]
    % plot[variable=\t,domain=87:62] (xyz spherical
    % cs:radius=1*\r,latitude=0,longitude=\t);
    
    %% estrellas
    \begin{scope}[on background layer]
%       \pgfmathsetseed{21}

%       % \tikzset{flechas/.style=yellow!90!white, line width=3pt,
%       %   tdplot_rotated_coords, ->, >= Latex}

%       \foreach \x in {1,...,3} {

%         % \pgfmathsetmacro{\lat}{int(acos(2*random()-1)-90)}
%         % \pgfmathsetmacro{\lat}{int(random(-1,1)*random(56,85))}
%         \pgfmathsetmacro{\lat}{0}
%         \pgfmathsetmacro{\lon}{random(0,360)}

% %        \ifthenelse{\lat=0}
%  %       {}
%   %     {    
%         \draw[decoration={shape backgrounds,shape=star,shape
%           size=5pt}, decorate, tdplot_rotated_coords,fill=yellow]
%         plot[variable=\lo,domain=\lon-1:\lon+1]
%         ({\r*.98*cos(\lo)*cos(\lat)},
%         {\r*.98*sin(\lo)*cos(\lat)},{\r*.98*sin(\lat)});
% %
%         \draw[yellow!70!white, line width=2pt, tdplot_rotated_coords,
%         ->, opacity=.8, >=Latex, line join=bevel ]
%         plot[variable=\lo,domain=\lon+2:\lon+55]
%         ({\r*1*cos(\lo)*cos(\lat)},
%         {\r*1*sin(\lo)*cos(\lat)},{\r*1*sin(\lat)});
%    %   }
%     }
    \end{scope}
    


      
  \end{tikzpicture}
\end{document}


