\chapter{A modo de introducción}

El presente librito, o manual, es el resultado de unas lecciones
escritas para un curso dictado durante el año 2021; de ahí el
carácter, quizás un tanto pretencioso, de <<2\sptext{da} edición>> con
que quise distinguirlo.

Dichas lecciones, a su vez, reconocen una doble inspiración. La
primera de ellas llegó de la mano de uno de mis colegas docentes,
Gonzalo Ciaffone, quien un día inolvidable llamó mi atención sobre el
mágico objeto protagonista de estas páginas, perfectamente desconocido
por mí hasta ese momento. Tras varios días dedicados a regocijarme
considerando su sabia y delicada maravilla, sentí la necesidad de
aplicarme a entender cómo reproducirlo: en otras palabras, a
entenderlo cabalmente. Para ello tuve primero que descubrir un
lenguaje de programación que me permitiera pensar y expresar la
construcción del reloj de Sol digital; afortunadamente lo hallé en
\openscad, el cual aprendí a chapucear primero, y a ejercer luego con
mayor solvencia, mientras resolvía el reloj.

La segunda inspiración de las lecciones es más fácil de declarar, aun
cuando resulta mucho más misteriosa: soy docente, y siento la
invencible necesidad de enseñar todo lo que aprendo.

El formato dialogado y con una cierta aspiración de novela que elegí
para el desarrollo de las lecciones y que mantuve para este manual me
pareció apropiado y divertido cuando urdí su inicio; con el correr de
las páginas comprobé luego que también me ayudaba a llevar adelante la
propia exposición del contenido teórico. No pocas fueron las
encrucijadas en las cuales el diálogo entre las protagonistas fue el
que me permitió descubrir la única manera de avanzar a través de
asuntos necesariamente difíciles de exponer. Espero que este formato
resulte también grato para el lector.

Otra virtud surgida de la escritura de las lecciones pude comprobarla
en el código producido durante la misma: la versión que obtuve
mientras las preparaba resultó mucho mejor que la que me permitió
imprimir mi primer reloj de Sol digital. Comparando ambos códigos no
puedo atribuir la diferencia exclusivamente a una mejor comprensión y
dominio del lenguaje de \openscad: no lo había ejercido tanto entre
uno y otro. Tengo para mí que fue el diálogo entre Cecilia y Antonia
---las mentadas protagonistas del librito--- el que me permitió
reflexionar más y mejor acerca del objeto que estaba (o estábamos ya)
escribiendo, lo cual resultó en un texto saludablemente más terso,
claro y optimizado. Esta virtud no debiera tomarme por sorpresa,
después de todo: de manera muy clara puede leerse en uno de los
consejos que enriquecen el libro \emph{The Pragmatic
  Programmer}.\footnote{\emph{The Pragmatic Programmer: Your Journey
    to Mastery, 20th Anniversary Edition}, Andrew Hunt y David Hurst
  Thomas, 2019.} En el tema 20 ---\emph{Debugging} (corrección de
errores)--- los autores proponen un proceder que denominan <<del
patito de hule>>, y que consiste esencialmente en obligarse a explicar
a alguien ---o, incluso, a \emph{algo}--- lo que se quiere resolver.

Quizá pueda parecer extraño a quien no escribe regularmente que el
solo hecho de poner en palabras una cuestión ayude a entenderla mejor,
o incluso a resolverla; sin embargo, es una realidad conocida desde
hace tiempo. Se dice que el propio Montaigne, en el siglo
\textsc{xvi}, ya había afirmado que nada aclara tanto las ideas como
ponerse a escribirlas.\footnote{No pude encontrar la referencia
  precisa; la noticia me llega de Paul Graham quien en su libro
  \emph{On Lisp} (Prentice Hall, 1993) la trae en la página 2.}

Atribuí al principio de esta introducción el origen inmediato de estas
páginas a un curso dictado durante el año 2021. No sería honesto de mi
parte dejar de confesar que su éxito resultó, cuando menos,
discutible: de los 15 participantes que se sintieron convocados por la
propuesta sólo dos me acompañaron hasta el final. Esta abrumadora
deserción seguramente se debe a una suma de causas variadas, aun
cuando la mayoría de ellas, ¡ay de mí!, sin duda deben reconocerme
como culpable. En cualquier caso me gustaría ofrecer en mi defensa la
frase de Séneca que uno de los dos alumnos que decidió terminar el
curso eligió para adornar su reloj de Sol digital, y que yo consideré
luego apropiado estampar al frente de los capítulos de esta segunda
edición: <<\emph{Non est ad astra mollis e terris via}>>: No hay un
camino fácil de la tierra a las estrellas. Serás tú, querido e
improbable lector, quien decida si esta inscripción es una bienvenida
o una amenaza.

Por último, me siento también en la obligación de sumarme al consejo
universal: si alguien pretende aprender a programar, no deberá
limitarse a leer, sino lanzarse a escribir. El aprendizaje de
cualquier idioma exige su ejercicio: los lenguajes de programación no
se encuentran al margen de esta necesidad. Es por esto por lo que te
aconsejo que escribas todos los textos que tu curiosidad te sugiera a
medida que avanzas a través de los capítulos, sin miedo frente a la
pantalla en blanco ni a los errores: El único \emph{bug} incorregible
es el que no se escribe.


\vfill

\section{Convenciones}

\subsection{Números de línea}

A fin de facilitar las referencias dentro del texto, se añaden números
de línea al código informático en su margen izquierdo:

\begin{lstlisting}
$fn = 200;

sphere(r=10);
cube([25,10,20]);
cylinder(h=15, r=5);
\end{lstlisting}%$

Por otra parte, cuando el código consta de una sola línea, tal
numeración se omite:

\begin{lstlisting}[numbers=none]
cube([20,30,40]);
\end{lstlisting}

%\newpage

\subsection{Prioridad de los operadores aritméticos}

\openscad{} respeta la prioridad usual de los operadores aritméticos;
de esta forma, una expresión como
  \begin{lstlisting}[numbers=none]
1+2*3-4    
  \end{lstlisting}
se interpreta como si estuviera escrita así:
  \begin{lstlisting}[numbers=none]
1+(2*3)-4    
  \end{lstlisting}
y, por lo tanto, devuelve ``\texttt{3}'' como resultado.



%%% Local Variables:
%%% mode: latex
%%% TeX-master: "../libro"
%%% End:
