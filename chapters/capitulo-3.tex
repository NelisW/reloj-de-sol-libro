\chapter{La idea}

\lettrine[ante=\raisebox{-2ex}{\LARGE ---¿},lines=2]{V}{os viste} los
relojes de Sol, ¿no?  ---empezó Antonia, y Cecilia asintió levemente,
resignada ya hacía tiempo a que le hiciera preguntas que suponían en
ella ignorancia acerca de los asuntos más elementales de la
Astronomía---. Pues bien ---pro\-si\-guió---, diseñé uno que aprovecha
el mismo principio, pero en lugar de indicar la hora `analógicamente'
mediante la sombra de un gnomón, la señala dejando pasar la luz por
huecos dispuestos de manera tal que sólo los atraviese limpiamente
cuando el Sol se encuentra a una altura determinada sobre el
horizonte.

Antonia, una vez más, había caído en uno de sus defectos favoritos:
priorizar la fluidez sintáctica y elegante brevedad superficial de un
período antes que la claridad y expresividad de su contenido. Cecilia
arqueó las cejas, esperando que, al dejarla hablar, lentamente fuera
quedando más o menos en claro su idea.

---La idea no es mía ---continuó Antonia---; Kenneth Falconer demostró
matemáticamente en 1987 que existe un objeto fractal cuyas sombras, si
es iluminado desde distintas direcciones cualesquiera, responde a un
diseño previo arbitrario.\footnote{Kenneth Falconer, \emph{Fractal
    Geometry: Mathematical Foundations and Applications}, 2nd Edition
  (2003), John Wiley \& Sons Ltd.} O sea, si elijo un conjunto
arbitrario de direcciones y sombras vinculadas uno a uno, existe
siempre un objeto fractal que garantiza la formación de cada sombra al
ser iluminado desde la dirección que le corresponde.

A Cecilia le pareció que empezaba a comprender. Lo sintió en todo su
cuerpo; siempre le pasaba lo mismo: las ideas nuevas, raras e
inesperadas le producían una felicidad e inquietud físicas. En esos
momentos se alegraba de haber elegido la Ciencia como profesión.

---Este coso ---de pronto Antonia abandonaba su pretensiosa elegancia
verbal para caer en un vocabulario preescolar, mientras le pasaba a
Cecilia el objeto misterioso a fin de que lo pudiera examinar más de
cerca--- puede considerarse el reverso del objeto fractal de Falconer:
en lugar de expresarse con sombras, lo hace con luz. Bueno, por
supuesto, tampoco es un fractal, ni deja pasar la luz de manera
distinta a cada instante. Supongo que tal objeto sería imposible de
realizar en la práctica. En todo caso, miralo y decime si no es
hermoso.

Cecilia pensó que Antonia tenía razón. El objeto que tenía entre sus
manos y que volvía y revolvía en ellas era muy hermoso; tanto más que
aún no lo comprendía del todo y por lo tanto estaba cargado con todas
las promesas del misterio y las posibilidades de la geometría. <<Este
objeto>> ---pensó Cecilia--- <<será infinito mientras no lo comprenda
del todo>>. No obstante, sabía que la pulsión por entender la
vencería, y la finitud se cerniría sobre ese semicilindro de plástico
que sus dedos acariciaban.

---¿No me vas a preguntar cómo lo hice? ---Antonia hizo pucherito,
simulando contrariedad y sonriendo con los ojos.

Cecilia salió con una sonrisa de su ensimismamiento y se dispuso a
escuchar.\footnote{Antonia parece olvidar una inspiración más cercana
  en el tiempo, y más concreta: en el sitio
  \url{https://www.thingiverse.com/thing:1068443} se muestra y
  comparte un modelo que acicateara al autor de estas páginas para
  replicar desde cero, lo mejor posible, tan mágico objeto. (Nota del
  Editor)}


%%% Local Variables:
%%% mode: latex
%%% TeX-master: "../libro"
%%% End:
