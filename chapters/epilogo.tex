\chapter{Epílogo}
\label{sec:epilogo}

\lettrine[lines=2]{D}{ebo admitir que} me invade una sensación de
incómoda perplejidad. Si bien no puedo ocultar que la idea de
conversar con Antonia y Cecilia me atrae desde hace tiempo ---para no
mencionar, claro, la dulce caricia a mi vanidad que representa
aparecer en primera persona dentro de este manualcito---, lo cierto es
que no tengo la menor idea de lo que voy a decirles. Y, por supuesto,
tengo aun menos claro que es lo que ellas quieren decirme. Me preocupa
sobre todo Cecilia, a quien parece que no dejé precisamente de buen
humor en el último capítulo. Supongo que lo mejor será dejar en manos
de ambas el hilo de este epílogo: casi no hice otra cosa, por lo
demás, en los capítulos precedentes.

Pero ya mi perplejidad parece a punto de terminar: el sonido familiar
de pasos en la escalera de entrada al Observatorio interrumpe mi
soliloquio inicial. Como siempre que esos pasos anuncian la llegada de
una visita me levanto del sillón de madera del escritorio, aliso
maquinalmente los pliegues de mi camisa con las manos y me bajo de la
tarima que preside el aula con la mirada puesta en la puerta y una
sonrisa amistosa flotando en los labios.

No me sorprendió ver aparecer primero a Cecilia, seguida
inmediatamente por Antonia. Las pocas fotografías que había visto de
ellas me permitieron reconocerlas enseguida: los fotógrafos de
principios del siglo \textsc{xx} eran perfectos retratistas, a los que los
fraudulentos programas de procesamiento de imágenes actuales no habían
privado, como lo consiguen ahora plenamente, de la capacidad de
capturar con fidelidad el carácter y la belleza de estas dos mujeres.

Las dos se detuvieron en cuanto traspasaron el umbral de la puerta del
aula, fijando en mí sus miradas. Los tres nos observamos en silencio
unos instantes. Preveo que será Cecilia quien inicie la conversación.

---Hola, Luis. ¿Cómo estás? ---dijo, confirmando mi sospecha. Su voz
era firme. Tal vez demasiado: intuí en esa firmeza un tanto enfática
que estaba tan nerviosa como yo.

---Muy bien, ¿y ustedes? ---procuro parecer tranquilo; no me cuesta
mucho, tampoco: de pronto recuerdo que soy el autor de este manual y
nada puede ocurrir en él fuera de mi voluntad. Considero apropiado
evocar algunos recuerdos comunes---. ¡Tanto tiempo, Cecilia!  Vos
fuiste mi alumna: imposible olvidarlo. Solías sentarte en... ese banco
---señalo con seguridad, confiando en no equivocarme---.  A vos,
Antonia, no tuve el gusto de conocerte personalmente. Pero, ¿quién no
te conoce aquí, en Harvard? Es un verdadero placer.

El gesto de Antonia se ensombreció ligeramente: ¿habrá pensado que la
conozco por supuestas malas habladurías que corren sobre ella? Olvidé
que es muy susceptible y vulnerable. Procuraré tenerlo más en cuenta.

Cecilia, por su parte, recibió mis palabras con un gesto radiante de
satisfacción; dirigiéndose a su amiga preguntó, como quien lanza un
desafío:

---¿Ves, Antonia? ¡Luis nos conoce! Yo fui su alumna, y vos sos una
celebridad en esta institución. Ergo, existimos. El asunto queda
zanjado y creo que ya podemos retirarnos ---con\-clu\-yó con tono
marcadamente triunfal, y a punto de dar media vuelta. Siento un gran
alivio: el manual terminará sin mayores sobresaltos y en paz.

Pero ahora fue Antonia quien parecía querer estirar esta incómoda
situación un poco más:

---Cecilia, siempre fuiste muy ingenua. Es un milagro que lograras ser
una gran científica confiando así en los demás ---se detuvo un
instante, y continuó---: Aunque, pensándolo bien, una científica debe
dudar de la naturaleza y sus leyes, más que de las personas. Como sea,
¿no te das cuenta de que Luis sólo simula haber sido tu profesor, como
lo hizo en el capítulo \ref{cap:poquito-de-astronomia}?

Cecilia la miró entre sorprendida y ofuscada: había olvidado que
Antonia era difícil de convencer. Me temo que deberé continuar unas
cuantas páginas más.

Antes de que su amiga pudiera replicar, Antonia se apresuró a reforzar
su posición:

---A ver, ¿recordás alguna pregunta que le hayas hecho en clase?

En este punto siento que puedo intervenir a favor de Cecilia:

---Bueno, Antonia; si es por eso, podría decirse que no tuve alumnos
de ningún tipo: ¡casi nadie me hace preguntas en clase!

Cuando se me ocurrió esta idea me pareció graciosa: apenas escrita,
advierto que es más bien melancólica.

Cecilia giró lentamente su mirada en mi dirección; en sus ojos flotaba
una luz que parecía de venganza:

---A ver el gran profesor que sí existe: ¿Ahora resultará que él
también pertenece a un mundo de ensueño e irrealidad?

Las palabras de Cecilia me toman por sorpresa. ¿De dónde habían
salido? ¿Se convertirá este epílogo en una confesión de mal gusto, con
tintes psicológicos de segunda o tercera categoría? Debo evitarlo
como sea.

---Por supuesto que no ---respondo como sin darle importancia al
asunto, y sonrío con la mayor suficiencia que puedo: tal vez hasta
parecer grosero---. Como todo adolescente pasé por mi período de
solipsismo, desde ya ---me parece necesario re\-co\-no\-\mbox{cer---;}
pero lo superé satisfactoriamente.

Cecilia parecía interesada en mi reciente confesión:

---¿Y cómo hiciste para superarlo?

No era muy difícil entender su interés: seguramente querría aferrarse
a mi ``método'' para emplearlo ella misma en la demostración de su
propia existencia. Ante ese evidente y fútil intento, esta vez no tuve
que forzar mi sonrisa.

---Pues... supongo que como todos ---digo, tratando de evocar
recuerdos bastante lejanos, tanto en el tiempo como en el conjunto de
mis preocupaciones actuales---. Un poco por cansancio y un poco por
obligación: sentí que debía actuar en función de los datos a los que
me enfrentaba. Aun si sólo fuera una mente solitaria pensándose y
pensando un Universo fantasmal, esa misma mente me lo representaba de
una manera específica, a la cual sentía mi deber adherir y responder,
aunque fuera de manera reactiva; pero nunca negándolo ---concluyo, con
el mayor aplomo que puedo.

---¡Ajá! ---lanzó Cecilia desafiante---. Puedo invocar el mismo
principio, entonces: yo percibo el mundo y mi existencia en él; luego,
debo aceptar que existo y actuar en consecuencia.

---Hay una gran diferencia, Cecilia ---señalo sua\-ve\-men\-te---.  Yo puedo
ejercer, o en todo caso soñar que ejerzo, mi libre albedrío: vos no.

Cecilia empalideció ligeramente.

---¿Qué querés decir?  ---preguntó bajando el tono de voz.

---Pues... yo puedo hacer que digas, ahora, la palabra `elefante'
---prometo, sin pretender ser demasiado original.

---¡Imposible! ---se rebeló, pero enseguida añade: ---E\-le\-fan\-te.

Giró rápidamente el rostro en dirección a Antonia, desconcertada; pero
sólo encontró a su amiga devolviéndole la mirada con gesto de
tranquila resignación.

---Elefante. Elefante. Elefante ---repite tres veces. Cecilia se
desplomó en uno de los bancos del aula; cubriéndose el rostro con las
manos, susurró:

---Está bien; entendí. Ya basta, por favor.

De pronto me siento muy cruel; tal vez no era necesario ser tan
categórico. Pero, ¿qué podía hacer? Este manual debe terminar alguna
vez, después de todo. No es que esté justificándome, tampoco; no tengo
porqué hacerlo, lector.

Como sea, ahora no se me ocurre mejor cosa que invitar a Antonia, con
un gesto conciliador, a sentarse en otro banco mientras yo rodeo el
escritorio para ocupar su correspondiente sillón.

Busco las palabras para tratar de suavizar la
situación:

---Cecilia, seamos sensatos. Ustedes \emph{tienen} una existencia
real.

Los ojos de ambas se abrieron con sorpresa; con una mano en alto
interrumpí posibles objeciones:

---No me malinterpreten; no me refiero a su existencia en el marco de
este manualcito. Como bien dijo Antonia en el capítulo anterior,
ustedes aquí son mis personajes: bastante mal construidos, por
cierto. `Casi de cartón' fue la expresión que usaste, si no me
equivoco.

Tomé el silencioso asentimiento de Antonia con un movimiento de cabeza
como una invitación a continuar:

---Pero tienen una existencia propia innegable: las dos serán recordadas
por siempre mientras la Astronomía siga siendo una actividad vital de
investigación y cuestionamiento constantes, tanto del mundo que nos
rodea como de nuestras propias inquietudes y miserias: inquietudes por
conocer y entender, pero también miserias al relegar y negar las
capacidades de algunos grupos humanos, ya sea por motivos de género o
de cualquier otra índole ---me dejo llevar por un cierto tono
retórico que no quiero interrumpir; procuro que mi voz sostenga una
contenida elevación apropiada al final del período que ya presiento---.
Ustedes existieron y existirán por siempre: para mostrarnos lo que
debemos hacer, lo que somos capaces de hacer, y lo que no debemos
permitir que nadie haga nunca más.

Hago un silencio teatral y las miro esperando que transformen sus
rostros en soles radiantes de felicidad y gratitud. Sin embargo, noto
que la expresión de ambas adquiere un tono reconcentrado, y de cierto
interés y preocupación. De pronto siento que no sólo me miran, sino
que me observan y analizan.

Esta vez fue Antonia quien retomó el hilo de la con\-ver\-sa\-ción:

---Lo que nos contaste ya lo sabemos, Luis: Nosotras no existimos en
este manual, pero fuimos grandes astrónomas y seremos recordadas por
siempre.

Tras intercambiar una mirada con Cecilia, quien pareció asentir
tácitamente a un pensamiento común, bajó la vista como si buscara la
mejor manera de continuar; inmediatamente agregó, clavando en mí su
mirada:

---¿Por qué nos trajiste aquí? ---y en su voz vibraba una nota de
delicada ternura, como quien habla con un niño a quién se desea hacer
sentir comprendido.

Me siento raro; es como si me hubieran descubierto en un delito cuya
naturaleza ignoro. Sólo alcanzo a repreguntar:

---¿Cómo por qué?  Supongo que me pareció una buena idea. No
sé... Algo gracioso. Ya conocen una de mis citas favoritas: `Toda
labor intelectual es, en última instancia, humorística'.

Sé muy bien que no es ésa la respuesta que esperan; alzando las cejas
las invito a continuar, no sin cierta aprensión.

Cecilia suspiró con fuerza, e intervino con una expresión de paciencia
contenida:

---Vos sabés que ésa no es la única razón. Repetiré la conclusión
anterior, para ser más clara: Antonia y yo no tenemos existencia aquí,
pero vivimos hace tiempo en Harvard y seremos recordadas por siempre
---e hizo silencio, mientras ambas me miraban como quien sólo espera
una conclusión evidente, la cual debo confesar que se me escapa. Sólo
puedo alzar los hombros, admitiendo mi perfecta incomprensión.

Antonia se mordió ligeramente los labios, y casi a su pesar agregó con
suavidad:

---Por otra parte, vos sí existís en el marco de este manual.

Sólo ahora entiendo lo que querían decirme; es un baldazo de agua
fría, tan repentino y violento como un relámpago. Me había propuesto
no convertir este epílogo en una confesión psicológica de segunda
categoría, y aquí me encuentro enfrentado por mis propios personajes a
la melancólica e innegable realidad: yo no seré recordado, ni dejaré
huella alguna permanente en la memoria de la humanidad. En cierto
sentido profundo y verdadero, son ellas quienes tienen una existencia
mucho más real que yo.

\label{pag:epilogo-2}
Me desplomo lentamente contra el respaldo del sillón de madera del
escritorio. Mi mirada se pierde, ausente, a través de las ventanas del
fondo del aula. <<¿Para esto las hice venir?>> ---me pregunto---.
<<¿Era necesario todo un epílogo que ya cuenta con
\pagedifferenceplusone{sec:epilogo}{pag:epilogo-2} páginas sólo como
torpe exégesis de los suficientes versos de Borges que no puedo dejar
de copiar cada vez que tengo la oportunidad de hacerlo..?>>
\settowidth{\versewidth}{La meta es el olvido.}
\PlainPoemTitle
\PoemTitle*{Un poeta menor}
\begin{verse}[\versewidth]
La meta es el olvido.\\
Yo he llegado antes.
\end{verse}

\vspace{1em}
\attrib{\emph{Quince monedas}, Jorge Luis Borges (1975).} %, en \emph{La rosa profunda}.

\vspace{1em}

El sabor de esos versos, tantas veces repetidos, vuelve a recordarme
la miel y la resignación. En medio de esa dulce melancolía casi me
olvido de mis compañeras, hasta que mis ojos tropiezan
inadvertidamente con los de Cecilia: me parece ver en ellos una luz
apagada de pena. ¿Será ella, ahora, quien se sienta mortificada por
haber sido tal vez demasiado cruel? ¿Deberé tolerar unas palabras de
consuelo por parte de mis propios personajes?  Eso sería demasiado
patético.

Fue Cecilia quien comenzó; no puedo decir que me sorprenda.

---La labor de un profesor es muy importante ---declaró con tono suave
y seguro---. Tiene en sus manos la capacidad de despertar en las
personas la curiosidad, el anhelo por conocer, por superarse.

Como docente me resulta imposible dejar de ver el reverso de la trama
del incipiente discurso de Cecilia; supongo que fue mi gesto de
desagrado lo que la detuvo en seco. Dirigió a su amiga una mirada que
era un pedido indisimulado de auxilio.

Antonia se acomodó mejor en su pupitre, mientras parecía buscar las
palabras adecuadas:

---Lo que queremos decir, Luis ---dijo con cautela y cierta
len\-ti\-tud---, es que hay muchas maneras de perdurar en la memoria
de los demás.

---Claro: se puede lograr siendo una gran científica cuyas obras se
impriman, divulguen, expliquen y analicen, o siendo un profesor en una
escuela secundaria ---interrumpo mordazmente, sin sentirme siquiera de
humor para conservar al menos una apariencia superficial de buena
educación.

Cecilia y Antonia apretaron los labios; parecen sentir que me estoy
poniendo difícil. Y bueno; ellas quisieron venir, después de
todo. Podrían haberse quedado tranquilamente en el capítulo
\ref{sec:texto}, como era mi intención original. Ahora que se
embromen.

Cecilia se agitó ligeramente, recobrando su aplomo de siempre. Eso me
gusta: supongo que no me espetará más lugares comunes.

---Luis, no seas injusto. No sólo con vos: tampoco con los
demás. Varios de tus alumnos tienen palabras muy elogiosas respecto a
vos ---aseguró, y con tono que descartaba toda objeción, agregó---: No
vas a sugerir que son falsos; eso sí sería muy cruel, sobre todo para
con ellos. La modestia es una virtud, pero la desconfianza está muy
cerca del insulto.

Las palabras de Cecilia me descolocan, debo admitirlo. Trato de
advertir en ellas un intento de manipulación, pero el vigor con que
las expresó de alguna manera me convencen de su sinceridad. Me levanto
del sillón y paseo por el aula, entre el escritorio y los bancos:
señal inequívoca de que necesito pensar.

De pronto siento que lo encontré:

---Ok. Acepto los elogios y su sinceridad ---considero más convincente
empezar reconociéndolo---. Pero, ¿qué significa exactamente eso?
---hago una pausa teatral para dar más fuerza a lo que sigue---. ¿Por
qué me elogian? ¿Porque soy un buen profesor?  ¿Porque soy una buena
persona? ¿Porque les caigo bien? ¿Porque...?

Esta vez fui yo quien se detuvo bruscamente al ver el gesto de ambas
amigas. Sí; tal vez fui demasiado lejos. Ahogo un suspiro mientras me
apoyo en la baranda de la escalera de acceso a la cúpula.

Los tres mantenemos un silencio pesado; nuestras miradas ya no se
cruzan. Mis ojos advierten sin interés alguno los rayos de Sol sobre
las baldosas del aula: no me resultan cálidos, ni brillantes, ni nada.

Vuelven a oírse pasos en la escalera: esta vez son veloces,
inquietos. El timbre del recreo acababa de sonar; seguramente es un
alumno en una visita fugaz.

Es Lucy; una sonrisa acude sin dificultad a mi rostro. Fue alumna mía
hace muy poco: compartí con ella, como con todos, inquietudes y
problemas astronómicos y de índole diversa. Subió los escalones de dos
en dos; estaba visiblemente apurada. No había alcanzado la puerta y ya
me saludaba con entusiasmo y alegría:

---¡Luissss!

Me extraña que pase al lado de Cecilia y Antonia como si no las viera,
dándoles olímpicamente la espalda: Lucy fue siempre muy educada. De
todas formas, no me da tiempo siquiera para presentarlas:

---Estuve en una clase muy aburrida ---soltó con tono vibrante y
veloz---. Se me ocurrió una manera muy loca de generar números con una
fórmula; mirá.

Puso ante mis ojos una hoja de carpeta cargada de símbolos y
tachaduras, que aprovechaban todo su espacio sin respetar los
euclídeos caminos de sus horizontales renglones. Traté de entenderlos,
mientras ella se transformaba en un volcán de explicaciones y
aclaraciones. Tomé en mis manos esa hoja: era desplolija, secreta,
redundante, inconclusa, oscura, titubeante, explosiva. Era, en una
palabra, hermosa.

El timbre volvió a sonar urgentemente. Lucy recuperó su hoja de entre
mis manos y con un sonoro <<¡Gracias, Luis! ¡Vuelvo más tarde!>> se
dirigió velozmente a la puerta. Apenas tuve tiempo de preguntar
inocentemente:

---¿Para qué es la fórmula? ¿Querés que te ayude en algo?

Lucy dio media vuelta, extrañada. Desde el umbral de la puerta
respondió:

---No, nada; es algo que se me ocurrió y te lo quería
mostrar. ¡Chauuuu..! ---y bajó por la escalera con la misma rapidez
con la que subió, sin tampoco despedirse ahora de Antonia y Cecilia.

Cuando el sonido de los pasos de Lucy se perdió a lo lejos, dirigí a
ellas la mirada. Estaban sonriéndome con ternura; en sus ojos flotaba
una luz de íntima satisfacción. Mientras me encontraba en el reflejo
de esos ojos repasé mentalmente lo ocurrido: una alumna había subido
al Observatorio a mostrarme un personal desarrollo matemático tal vez
sin sentido y para el cual no tenía otra motivación que la pasión por
realizarlo.

De pronto me sentí como vencido; debo haberme ruborizado, incluso.

Quise decir algo, pero no supe qué. No fue necesario; Cecilia y
Antonia se levantaron lentamente, sin hacer el menor ruido. Se
dirigieron hacia la puerta, desde donde se despidieron moviendo
levemente sus manos en alto. Sus sonrisas seguían diciéndome desde el
umbral que ellas, finalmente, tenían razón.

No oigo sus pasos a medida que se alejan por la escalera. Ya no me
extraña ni me deja de extrañar.

---Adiós, queridas amigas.

---Y... muchas gracias.

Nuevamente a solas, mis ojos recorren con gratitud y cariño el
familiar espacio del aula vacía. Los pupitres, apenas ordenados, son
otras tantas promesas que florecerán, sin duda, el año que viene.

Encuentro en los rayos del Sol que entran al aula toda la luz y la
calidez que hasta hace un momento habían perdido. Las ventanas y los
bancos sólo permiten que algunos de esos rayos toquen las baldosas,
escribiendo en ellas con símbolos secretos que ningún algoritmo será
capaz de repetir.

Suspiro profundamente. Estoy por volver al sillón de madera cuando una
chispa se enciende bruscamente en mi alma; salgo disparado hacia la
puerta mientras procuro que mi voz alcance a oírse:

---¡Esperen! ¡Se me acaba de ocurrir una idea...!

%\vspace{4em}

\vfill

\begin{center}
  \emph{FIN}
\end{center}

\vfill

%%% Local Variables:
%%% mode: latex
%%% TeX-master: "../libro"
%%% End:
